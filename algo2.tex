\documentclass[12pt]{article}
\usepackage[utf8]{inputenc}
\usepackage[english, greek]{babel}
\usepackage[LGR]{fontenc}
\usepackage{amsmath}
\usepackage{mathtools}
\usepackage{enumitem}
\usepackage{listings}
\usepackage{slashbox}
\usepackage{multirow}
\usepackage{pgfplots}
\usepackage{enumerate}
\usepackage{gensymb}
\usepackage{amssymb}
\usepackage[top=2cm,bottom=2cm, left=2.5cm,right=2.5cm]{geometry}
\usepackage{graphicx}
\usepackage{eurosym}
\usepackage{enumitem}
\usepackage{caption}
\usepackage{moreenum}
\usepackage{forest}
\usepackage[margin=5 pt]{subcaption}
\useshorthands{;}
\defineshorthand{;}{?}
\usepackage{float}
\usepackage{tkz-fct}
\usetkzobj{all}
\graphicspath{ {images/} }
\usepackage{tikz}
\usetikzlibrary{shapes.geometric, arrows}
\DeclarePairedDelimiter\ceil{\lceil}{\rceil}
\usepackage[colorinlistoftodos]{todonotes}
\usepackage{xcolor}
\tikzstyle{startstop}=[rectangle, rounded corners,minimum width=3cm, minimum height=1cm,text centered, draw=black, fill=red!20]
\tikzstyle{io} = [trapezium, trapezium left angle=70, trapezium right angle=110, minimum width=3cm, minimum height=1cm, text centered, draw=black, fill=blue!10]
\tikzstyle{process} = [rectangle, minimum width=3cm, minimum height=1cm, text centered, draw=black, fill=yellow!30]
\tikzstyle{decision} = [diamond, minimum width=3cm, minimum height=1cm, text centered, draw=black, fill=green!30]
\tikzstyle{arrow} = [thick,->,>=stealth]
\definecolor{vgreen}{RGB}{104,180,104}
\definecolor{vblue}{RGB}{49,49,255}
\definecolor{vorange}{RGB}{255,143,102}
\lstset{
      language=Verilog,
    basicstyle=\small\selectlanguage{english}\ttfamily,
   % keywordstyle=\color{vblue},
    identifierstyle=\color{black},
    commentstyle=\color{vgreen},
    numbers=left,
    numberstyle=\tiny\color{black},
    numbersep=10pt,
    tabsize=8,
    moredelim=*[s][\colorIndex]{[}{]},
    literate=*{:}{:}1
}


\makeatletter
\newcommand*\@lbracket{[}
\newcommand*\@rbracket{]}
\newcommand*\@colon{:}
\newcommand*\colorIndex{%
    \edef\@temp{\the\lst@token}%
    \ifx\@temp\@lbracket \color{black}%
    \else\ifx\@temp\@rbracket \color{black}%
    \else\ifx\@temp\@colon \color{black}%
    \else \color{black}%
    \fi\fi\fi
}
\makeatother

\usepackage{trace}
\begin{document}
\begin{titlepage}

\newcommand{\HRule}{\rule{\linewidth}{0.5mm}} 

\center 

\textsc{\LARGE Εθνικό Μετσόβιο Πολυτεχνείο}\\[1.5cm] 
\textsc{\large ΗΛΕΚΤΡΟΛΟΓΩΝ ΜΗΧΑΝΙΚΩΝ ΚΑΙ ΜΗΧΑΝΙΚΩΝ ΥΠΟΛΟΓΙΣΤΩΝ}\\[1.5cm] 
\includegraphics[scale=0.4]{ntua.png}\\[2cm] 
\textsc{\Large Αλγόριθμοι και πολυπλοκότητα}\\[1cm] 
\hfill \break

\HRule \\[0.4cm]
{ \huge \bfseries 2η Γραπτή Σειρά Ασκήσεων }\\[0.4cm] 
\HRule \\[1.5cm]
 
\hfill \break
\begin{minipage}{0.4\textwidth}
\begin{flushleft} \large
\centering  Γαλανάκη Σοφία\\
\centering 03115060
\end{flushleft}
\hfill \break
\hfill \break
\hfill \break
\hfill \break

\end{minipage}\\[2.5cm]

{\large \today}\\[2cm] 

\vfill 

\end{titlepage}
\begin{itemize}[label=$\blacktriangleright$]
\item \textbf{Άσκηση 2: Τηλεπαιχίδι}\\
\hfill \break
Αρχικά, αποθηκεύουμε τις τιμές $f_j$ για καθένα από τα σφυριά σε έναν πίνακα έστω $F$. Επίσης αποθηκεύουμε σε εναν άλλο πίνακα έστω Β $structs$ με δύο πεδία, ένα για το ποσό που κερδίζουμε σε περίπτωση που σπάσει και ένα για την τιμή $p_i$. Ύστερα, εφαρμόζουμε αλγόριθμο ταξινόμησης στους δύο αυτούς πίνακες (O($nlogn)$). Πιο συγκεκριμένα, ταξινομούμε κατά αύξουσα σειρά τον πίνακα με τα σφυριά και κατά φθίνουσα τον πίνακα με τα κουτιά με βάση το ποσό που κερδίζουμε σε περίπτωση που το σπάμε. \\
Προσπελαύνουμε τον ταξινομημένο πίνακα κουτιών και κάνοντας διαδική αναζήτηση στον ταξινομημένο πίνακα σφυριών βρίσκουμε το σφυρί της μικρότερης δυνατής δύναμης το οποίο μπορεί να σπάσει το κουτί.\\ Φυσικά η διαδικασία αυτή επαναλαμβάνεται για κάθε στοιχείο του πίνακα των κουτιών, που σημαίνει ότι χρειαζόμαστε χρόνο Ο($nlogn$).\\
\item \textbf{Άσκηση 3: Αναμνηστικά}\\
Αρχικά, βρίσκουμε τη βέλτιστη λύση.\\
Σκεφτόμαστε ότι για κάθε χώρα θέλουμε να βρούμε ένα αναμνηστικό ώστε συνολικά το κόστος να μην ξεπερνάει το $C$ και να μεγιστοποιείται η συναισθηματική αξία.\\
Συνεπώς, εκφράζουμε το πρόβλημα αυτό αναδρομικά ως εξής:
\begin{center}
\begin{equation}
 Value[i,C]=
 \begin{cases}
    \max\limits_{1 \leq j \leq k_i}\{p_{i,j}+Value[i-1,C-c_{i,j}]\}, & \text{$i\geq2,\ C-c_{i,j}\geq0$}.\\
    -\infty, & \text{$i\geq2,\ C-c_{i,j}\leq0$}.\\
    \max\limits_{1 \leq j \leq k_1}\{p_{1,j}| c_{1,j}\leq C\}, & \text{$i=1$}.\\
  \end{cases}
\end{equation}
\end{center}

Αν φτιάξουμε το δέντρο αναδρομής για κάποιο παράδειγμα εισόδου θα δούμε ότι πολλές κλήσεις της αναδρομής και συνεπώς πολλοί υπολογισμοί επαναλαμβάνονται. Επομένως, εφαρμόζουμε δυναμικό προγραμματισμό κρατώντας κάποιους από τους υπολογισμούς ώστε να μη χρειαστεί να επαναληφθούν.\\
Δημιουργούμε έναν πίνακα με $n$ γραμμές, όσες και οι χώρες που θα επισκεφτούμε και με $c+1$ στήλες (από 0 έως την τιμή $C$).\\
Αρχικά, θα αποθηκεύσουμε για την πρώτη χώρα που θα επισκεφτούμε το αναμνηστικό με την μεγαλύτερη συναισθηματική αξία που να μην ξεπερνάει κάθε τιμή του κόστους. Για παράδειγμα στη γραμμή που αντιστοιχεί στη χώρα 1 και στο κόστος 4 θα τοποθετήσουμε το αναμνηστικό εκείνο που κοστίζει το πολύ 4 και συγχρόνως έχει τη μεγαλύτερη αξία από αυτά που κοστίζουν το πολύ 4.\\
Κατ' αυτόν τον τρόπο γεμίζουμε την πρώτη σειρά του πίνακα. Για τη δεύτερη σειρά, παρατηρούμε από τον τύπο της αναδρομής (1), ότι χρειαζόμαστε τις τιμές που υπάρχουν στην πρώτη σειρά. Για την τρίτη σειρά, πάλι από τον τύπο της αναδρομής, βλέπουμε ότι χρειαζόμαστε τις τιμές που υπάρχουν στην δεύερη σειρά κ.ο.κ..\\
Επομένως, για να συπληρωθεί η $i$-οστή σειρά του πίνακα χρειαζόμαστε χρόνο $(C+1)k_i$.\\
Άρα, συνολικά για τη συμπλήρωση όλου του πίνακα χρειαζόμαστε χρόνο $(C+1)\sum_{i=1}^{n}(k_i)$. Άρα η πολυπλοκότητα είναι Ο($C\sum_{i=1}^{n}(k_i)$).
\item \textbf{Άσκηση 5: Πομποί και δέκτες}\\
\hfill \break
Αρχικά, βρίσκουμε βέλτιστη λύση για το πρόβλημα. Η απόφαση που πρέπει να πάρουμε για την $i$-οστή κεραία είναι αν είναι πομπός ή όχι. Κατασκευάζουμε , λοιπόν, την αναδρομική σχέση ως εξής:
\begin{equation}
 S[i,N/2]=
 \begin{cases}
  	\min\{S[i-1,N/2-1]+T_i,S[i-1,N/2]+R_i\}, & \text{$i\geq2, i-1\leq N$}.\\
  	{S[i-1,N/2]+R_i}, & \text{$i\geq2, i-1>N$}.\\
  	T_1, & \text{$i=1, 	N/2=1$}.\\
  	\infty, & \text{$i\geq2, N/2=0$}.\\
  \end{cases}
\end{equation}
Η υλοποίηση της παραπάνω σχέσης σύμφωνα με το δυναμικό προγραμματισμό γίνεται ως εξής:\\

\end{itemize}
\end{document}
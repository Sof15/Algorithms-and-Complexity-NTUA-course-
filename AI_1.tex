\documentclass[12pt]{article}
\usepackage[utf8]{inputenc}
\usepackage[english, greek]{babel}
\usepackage[LGR]{fontenc}
\usepackage{amsmath}
\usepackage{mathtools}
\usepackage{enumitem}
\usepackage{listings}
\usepackage{slashbox}
\usepackage{multirow}
\usepackage{pgfplots}
\usepackage{enumerate}
\usepackage{gensymb}
\usepackage{amssymb}
\usepackage[top=2cm,bottom=2cm, left=1.5cm,right=1.5cm]{geometry}
\usepackage{graphicx}
\usepackage{enumitem}
\usepackage{caption}
\usepackage{moreenum}
\usepackage{forest}
\usepackage[margin=5 pt]{subcaption}
\useshorthands{;}
\defineshorthand{;}{?}
\usepackage{float}
\usepackage{tkz-fct}
\usetkzobj{all}
\graphicspath{ {images/} }
\usepackage{tikz}
\usetikzlibrary{shapes.geometric, arrows}
\DeclarePairedDelimiter\ceil{\lceil}{\rceil}
\usepackage[colorinlistoftodos]{todonotes}
\usepackage{xcolor}
\tikzstyle{startstop}=[rectangle, rounded corners,minimum width=3cm, minimum height=1cm,text centered, draw=black, fill=red!20]
\tikzstyle{io} = [trapezium, trapezium left angle=70, trapezium right angle=110, minimum width=3cm, minimum height=1cm, text centered, draw=black, fill=blue!10]
\tikzstyle{process} = [rectangle, minimum width=3cm, minimum height=1cm, text centered, draw=black, fill=yellow!30]
\tikzstyle{decision} = [diamond, minimum width=3cm, minimum height=1cm, text centered, draw=black, fill=green!30]
\tikzstyle{arrow} = [thick,->,>=stealth]
\definecolor{vgreen}{RGB}{104,180,104}
\definecolor{vblue}{RGB}{49,49,255}
\definecolor{vorange}{RGB}{255,143,102}
\lstset{
      language=Verilog,
    basicstyle=\small\selectlanguage{english}\ttfamily,
   % keywordstyle=\color{vblue},
    identifierstyle=\color{black},
    commentstyle=\color{vgreen},
    numbers=left,
    numberstyle=\tiny\color{black},
    numbersep=10pt,
    tabsize=8,
    moredelim=*[s][\colorIndex]{[}{]},
    literate=*{:}{:}1
}


\makeatletter
\newcommand*\@lbracket{[}
\newcommand*\@rbracket{]}
\newcommand*\@colon{:}
\newcommand*\colorIndex{%
    \edef\@temp{\the\lst@token}%
    \ifx\@temp\@lbracket \color{black}%
    \else\ifx\@temp\@rbracket \color{black}%
    \else\ifx\@temp\@colon \color{black}%
    \else \color{black}%
    \fi\fi\fi
}
\makeatother

\usepackage{trace}
\tikzset{
  my blue box/.style={fill=gray!20, rectangle, rounded corners=2pt},
}
\forestset{
  my blue label/.style={
    label={[my blue box]above:#1},
   % s sep+=10pt,
  }
}
\begin{document}
\begin{titlepage}

\newcommand{\HRule}{\rule{\linewidth}{0.5mm}} 

\center 

\textsc{\LARGE Εθνικό Μετσόβιο Πολυτεχνείο}\\[1.5cm] 
\textsc{\large ΗΛΕΚΤΡΟΛΟΓΩΝ ΜΗΧΑΝΙΚΩΝ ΚΑΙ ΜΗΧΑΝΙΚΩΝ ΥΠΟΛΟΓΙΣΤΩΝ}\\[1.5cm] 
\includegraphics[scale=0.4]{ntua.png}\\[2cm] 
\textsc{\Large Τεχνητή Νοημοσύνη}\\[1cm] 
\hfill \break

\HRule \\[0.4cm]
{ \huge \bfseries 1η Γραπτή Σειρά Ασκήσεων }\\[0.4cm] 
\HRule \\[1.5cm]
 
\hfill \break
\begin{minipage}{0.4\textwidth}
\begin{flushleft} \large
\centering  Γαλανάκη Σοφία\\
\centering 03115060
\end{flushleft}
\hfill \break
\hfill \break
\hfill \break
\hfill \break

\end{minipage}\\[2.5cm]

{\large \today}\\[2cm] 

\vfill 

\end{titlepage}
\begin{itemize}[label=$\blacktriangleright$]
\item \textbf{Άσκηση 1}\\
\begin{enumerate}
\item Εκτέλεση αλγορίθμου \textlatin{Hill Climbing}\\
\begin{table}[H]
\begin{tabular}{cccc}
Μέτωπο Αναζήτησης & Κατάσταση & Παιδιά    & Ευριστική                                    \\
$<(s,9)>$      & $s$       & $<b,c,d>$ & $b\rightarrow6,c\rightarrow4, d\rightarrow5$ \\
$<(c,4)>$         & $c$       & $<h>$     & $h\rightarrow5$                              \\
\end{tabular}
\end{table}
Εκτέλεση αλγορίθμου \textlatin{Best First}\\
\begin{table}[H]
\begin{tabular}{ccccc}
Μέτωπο Αναζήτησης     & Κλειστό Σύνολο  & Κατάσταση & Παιδιά    & Ευριστική                                     \\
$<(s,9)>$             & $<>$           & $s$       & $<b,c,d>$ & $b\rightarrow6 ,c\rightarrow4, d\rightarrow5$ \\
$<(c,4),(d,5),(b,6)>$ & $<s>$         & $c$       & $<h>$     & $h\rightarrow5$                               \\
$<(d,5),(h,5),(b,6)>$ & $<s,c>$       & $d$       & $<h,i>$   & $h\rightarrow5, i\rightarrow3$                \\
$<(i,3),(h,5),(b,6)>$ & $<s,c,d>$     & $i$       & $<j>$     & $j\rightarrow3$                               \\
$<(j,3),(h,5),(b,6)>$ & $<s,c,d,i>$   & $j$       & $<g>$     & $g\rightarrow0$                               \\
$<(g,0),(h,5),(b,6)>$ & $<s,c,d,i,j>$ & $g$       &           &                                               \\
                      &                 &           &           &                                              
\end{tabular}

\end{table}
Εκτέλεση αλγορίθμου $Α^{*}$\\
\begin{table}[H]
\resizebox{\textwidth}{!}{%
\begin{tabular}{cccccc}
Μέτωπο Αναζήτησης               & Κλειστό Σύνολο        & Κατάσταση & Παιδιά    & $G$                                           & $F$                                           \\
$<(s,0)>$                       & $<>$                  & $s$       & $<b,c,d>$ & $b\rightarrow5, c\rightarrow2, d\rightarrow2$ & $b\rightarrow11, c\rightarrow6,d\rightarrow7$ \\
$<(c,6),(d,7),(b,11)>$          & $<s>$                 & $c$       & $<h>$   & $h\rightarrow8$               & $h\rightarrow13$              \\
$(d,7),(b,11),(h,13)>$   & $<s,c>$               & $d$       & $<h,i>$     & $h\rightarrow10,i\rightarrow8$                               & $h\rightarrow15,i\rightarrow11$                              \\
$<(b,11),(i,11),(h,13)>$        & $<s,c,d>$             & $b$       & $<e,k>$   & $e\rightarrow9, k\rightarrow7$                & $e\rightarrow14, k\rightarrow9$               \\
$<(k,9),(i,11),(h,13),(e,14)>$  & $<s,c,d,b>$           & $k$       & $<g,h>$   & $g\rightarrow18, h\rightarrow8$               & $g\rightarrow18, h\rightarrow13$              \\
$<(i,11),(h,13),(e,14),(g,18)>$ & $<s,c,d,b,k>$         & $i$       & $<j>$     & $j\rightarrow12$                              & $j\rightarrow15$                              \\
$<(h,13),(e,14),(j,15),(g,18)>$ & $<s,c,d,b,k,i>$       & $h$       & $<i,j>$   & $i\rightarrow11, j\rightarrow15$              & $i\rightarrow14, j\rightarrow18$              \\
$<(e,14),(j,15),(g,18)>$        & $<s,c,d,b,k,i,h>$     & $e$       & $<g>$     & $g\rightarrow18$                              & $g\rightarrow18$                              \\
$<(j,15),(g,18)>$               & $<s,c,d,b,k,i,h,e>$   & $j$       & $<g>$     & $g\rightarrow14$                              & $g\rightarrow14$                              \\
$<(g,14)>$                      & $<s,c,d,b,k,i,h,e,j>$ & $g$       &           &                                               &                                              
\end{tabular}%
}
\end{table}
\item Το πρόβλημα έχει 9 λύσεις που είναι οι ακόλουθες:
\begin{itemize}
\item $s\rightarrow b\rightarrow e\rightarrow g$\\
Συνολικό πραγματικό κόστος: 18
\item $s\rightarrow b\rightarrow k\rightarrow g$\\
Συνολικό πραγματικό κόστος: 18
\item $s\rightarrow b\rightarrow k\rightarrow h\rightarrow j\rightarrow g$\\
Συνολικό πραγματικό κόστος: 17
\item $s\rightarrow b\rightarrow k\rightarrow h\rightarrow i\rightarrow j\rightarrow g$\\
Συνολικό πραγματικό κόστος: 17
\item $s\rightarrow c\rightarrow h\rightarrow j\rightarrow g$\\
Συνολικό πραγματικό κόστος: 17
\item $s\rightarrow c\rightarrow h\rightarrow i\rightarrow j\rightarrow g$\\
Συνολικό πραγματικό κόστος: 17
\item $s\rightarrow d \rightarrow h\rightarrow j\rightarrow g$\\
Συνολικό πραγματικό κόστος: 19
\item $s\rightarrow d \rightarrow h\rightarrow i\rightarrow j\rightarrow g$\\
Συνολικό πραγματικό κόστος: 19
\item $s\rightarrow d \rightarrow i\rightarrow j \rightarrow g$\\
Συνολικό πραγματικό κόστος: 14
\end{itemize}
Από τα παραπάνω συμπεραίνουμε ότι η βέλτιστη λύση είναι η $s\rightarrow d \rightarrow i\rightarrow j \rightarrow g$ καθώς έχει το μικρότερο συνολικό πραγματικό κόστος.\\
Από τους παραπάνω αλγορίθμους μόνο ο $Hill \ Climbing$ δε βρίσκει λύση.\\
Το πρόβλημα του πρώτου αλγορίθμου είναι ότι αυτός χρησιμοποιεί μόνο τις ευριστικές τιμές και όπως φάνηκε εκ του αποτελέσματος, αυτές δεν είναι και τόσο καλές.\\
Επίσης οι αλγόριθμοι Α* και $best \ first$ δίνουν τη βέλτιστη λύση.\\
Στον γράφο που περιγράφει τις καταστάσεις του χώρου αναζήτησης παρατηρούμε ότι σε οποιαδήποτε κατάσταση και αν βρεθούμε υπάρχει κατευθυνόμενο μονοπάτι που να μας οδηγεί στην τελική κατάσταση ($g$). Επομένως κάθε αλγόριθμος θα βρει λύση ακόμα και αν αυτή δεν είναι η βέλτιστη.
\end{enumerate}
\item
\begin{enumerate}
\item Εφαρμόζοντας τον αλγόριθμο \textlatin{Minimax} προκύπτει το ακόλουθο δέντρο.\\
\hfill \break
\begin{forest}
for tree={circle,draw,l sep=1cm, s sep =0.2 cm,minimum size=1em},
[$4$, my blue label=1
    [$3$, my blue label=2
      [$4$ , my blue label=5
        [$4$, my blue label=11
        	[$6$, my blue label=23]
        	[$4$, my blue label=24]
        ]
        [$2$, my blue label=12, for tree={l sep=3cm}
        	[$2$, my blue label=25]
        	[$9$, my blue label=26]
        	[$6$, my blue label=27]
        ]
      ]
      [$3$ , my blue label=6
        [$2$, my blue label=13
        	[$3$, my blue label=28]
        	[$2$, my blue label=29]
        ]
        [$1$, my blue label=14, for tree={l sep=3cm}
        	[$1$, my blue label=30]
        ]
        [$3$, my blue label=15
        	[$5$, my blue label=31]
        	[$3$, my blue label=32]
        ]
	]
    ]
    [$1$, my blue label=3
      [$1$, my blue label=7
        [$1$, my blue label=16, for tree={l sep=3cm}
        	[$3$, my blue label=33]
        	[$1$, my blue label=34]
        	[$8$, my blue label=35]
        ]
	]
      [$5$, my blue label=8
        [$5$, my blue label=17
        	[$5$, my blue label=36]
        	[$6$, my blue label=37]
        ]
	]
  ] 
   [$4$, my blue label=4
      [$4$, my blue label=9
        [$4$, my blue label=18, for tree={l sep=3cm}
        	[$8$, my blue label=38]
        	[$4$, my blue label=39]
        	[$6$, my blue label=40]
        ]
        [$2$, my blue label=19
        	[$2$, my blue label=41]
        	[$9$, my blue label=42]
        ]
	]
      [$4$, my blue label=10
        [$4$, my blue label=20,for tree={l sep=3cm}
        	[$4$, my blue label=43]
        	[$9$, my blue label=44]
        ]
        [$3$, my blue label=21
        	[$5$, my blue label=45]
        	[$3$, my blue label=46]
        ]
        [$1$, my blue label=22, for tree = {l sep=3cm}
        	[$3$, my blue label=47]
        	[$1$, my blue label=48]
        	[$2$, my blue label=49]
        ]
	]
  ]
]
\end{forest}
\newpage
\item Εφαρμόζοντας τον αλγόριθμο ΑΒ προκύπτει το ακόλουθο δέντρο.
\\
\hfill \break
Αρχικά, πάμε στο αριστερότερο κλαδί και επισκεπτόμαστε τους κόμβους 23,24 με τιμές $6,4$ και άρα ο κόμβος 11 παίρνει την τιμή 4.\\
Ύστερα, επισκεπτόμαστε στο δεύτερο κλαδί τον κόμβο 25 με τιμή $2$ και επειδή το $min$ αυτού του κλαδιού δε μπορεί να είναι μεγαλύτερο από 2, ισχύει ότι το $min$ αυτό δε θα είναι σε καμία περίπτωση μεγαλύτερο από την τιμή 4 που δώσαμε στον κόμβο 11. Άρα και ο κόμβος 5 παίρνει την τιμή 4.\\
Στη συνέχεια επισκεπτόμαστε τους κόμβους 28,29 με τιμές 3,2 και βάζουμε στον κόμβο 13 την τιμή 2. Μετά επισκεπτόμαστε τον κόμβο 30 με τιμή 1 και βάζουμε στον κόμβο 14 την τιμή 1. Αντίστοιχα και για τους κόμβους 31,32 και 15. Ο τελευταίος παίρνει την τιμή 3. Επομένως ο κόμβος 6 παίρνει την τιμή 3 και ο 2 την τιμή 4.\\
Προχωράμε στο επόμενο υπόδεντρο και επισκεπτόμαστε τον κόμβο 33 με τιμή 3, οπότε ο κόμβος 16 θα έχει τιμή το πολύ 3. Αυτό σημαίνει ότι ο κόμβος 3 θα πάρει τιμή το πολύ 3. \\
Έπειτα επισκεπτόμαστε τους κόμβους 38,39,40 με τιμές 8,4,6 και βάζουμε στον κόμβο 18 την τιμή 4. Επισκεπτόμαστε τους κόμβους 41 με τιμή 2 και άρα ο κόμβος 19 θα έχει τιμή το πολύ 2 και επομένως ο κόμβος 9 θα πάρει την τιμή 4.\\
Επισκεπτόμαστε τους κόμβους 43,44 με τιμές 4,9 και βάζουμε στον κόμβο 20 την τιμή 4. Αυτό σημαίνει ότι κόμβος 10 θα έχει τιμή τουλάχιστον 4 και συνεπώς ο κόμβος 4 θα πάρει και αυτός τιμή 4.
\\
Τελικά, ο κόμβος 1 αποτιμάται με 4.\\
\hfill \break
\begin{forest}
for tree={circle,draw,l sep=1cm, s sep =0.2 cm,minimum size=1.6em},
[$4$, my blue label=1
    [$3$, my blue label=2
      [$4$ , my blue label=5
        [$4$, my blue label=11
        	[$6$, my blue label=23]
        	[$4$, my blue label=24]
        ]
        [$\leq2$, my blue label=12, for tree={l sep=3cm}
        	[$2$, my blue label=25]
        	[$9$, my blue label=26, for tree={fill=lightgray}]
        	[$6$, my blue label=27, for tree={fill=lightgray}]
        ]
      ]
      [$3$ , my blue label=6
        [$2$, my blue label=13
        	[$3$, my blue label=28]
        	[$2$, my blue label=29]
        ]
        [$1$, my blue label=14, for tree={l sep=3cm}
        	[$1$, my blue label=30]
        ]
        [$3$, my blue label=15
        	[$5$, my blue label=31]
        	[$3$, my blue label=32]
        ]
	]
    ]
    [$\leq3$, my blue label=3
      [$\leq3$, my blue label=7
        [$\leq3$, my blue label=16, for tree={l sep=3cm}
        	[$3$, my blue label=33]
        	[$1$, my blue label=34, for tree = {fill=lightgray}]
        	[$8$, my blue label=35, for tree = {fill=lightgray}]
        ]
	]
      [$$, my blue label=8,for tree={fill=lightgray}
        [$$, my blue label=17
        	[$5$, my blue label=36]
        	[$6$, my blue label=37]
        ]
	]
  ] 
   [$4$, my blue label=4
      [$4$, my blue label=9
        [$4$, my blue label=18, for tree={l sep=3cm}
        	[$8$, my blue label=38]
        	[$4$, my blue label=39]
        	[$6$, my blue label=40]
        ]
        [$\leq2$, my blue label=19
        	[$2$, my blue label=41]
        	[$9$, my blue label=42, for tree={fill=lightgray}]
        ]
	]
      [$\geq4$, my blue label=10
        [$4$, my blue label=20,for tree={l sep=3cm}
        	[$4$, my blue label=43]
        	[$9$, my blue label=44]
        ]
        [$$, my blue label=21, for tree = {fill=lightgray}
        	[$5$, my blue label=45, for tree = {fill=lightgray}]
        	[$3$, my blue label=46, for tree = {fill=lightgray}]
        ]
        [$$, my blue label=22, for tree = {l sep=3cm,fill=lightgray}
        	[$3$, my blue label=47,for tree ={fill=lightgray}]
        	[$1$, my blue label=48, for tree={fill=lightgray}]
        	[$2$, my blue label=49, for tree={fill=lightgray}]
        ]
	]
  ]
]
\end{forest}
\newpage
\item Αρχικά σειρά έχει να παίξει ο $max$. Η καλύτερη κίνηση που μπορεί να κάνει είναι να διαλέξει τον κόμβο 4. \\
Ύστερα, είναι η σειρά του $min$ να παίξει. Η καλύτερη κίνηση που μπορεί να κάνει είναι να διαλέξει αν θα πάει αριστερά (κόμβος 9 $\rightarrow$ $1^\eta$ περίπτωση) ή αν θα πάει δεξιά (κόμβος 10 $\rightarrow$ $2^\eta$ περίπτωση). \\
 \textbf{$1^\eta$ περίπτωση:}\\
Σειρά έχει ο $max$ και η καλύτερη κίνηση που μπορεί να κάνει είναι να πάει στον κόμβο 18.\\
Σειρά έχει ο $min$ και το καλύτερο που μπορεί να κάνει είναι να πάει στον κόμβο 39.\\
\textbf{ $2^\eta$ περίπτωση:}\\
Σειρά έχει ο $max$ και η καλύτερη κίνηση που μπορεί να κάνει είναι να πάει στον κόμβο 20.\\
Σειρά έχει ο $min$ και το καλύτερο που μπορεί να κάνει είναι να πάει στον κόμβο 43.\\
\end{enumerate}

\end{itemize}
\end{document}